\documentclass{classrep}
\usepackage[utf8]{inputenc}
\usepackage{color}
\usepackage{ltablex}
\usepackage{makebox}
\usepackage{amsmath}
\usepackage{hyperref}

\studycycle{Informatyka, studia dzienne, II st.}
\coursesemester{II}

\coursename{Komputerowe systemy rozpoznawania}
\courseyear{2011/2012}

\courseteacher{dr inż. Arkadiusz Tomczyk}
\coursegroup{poniedziałek, 10:15}

\author{
  \studentinfo{Michał Janiszewski}{169485} \and
  \studentinfo{Mariusz Łucka}{169493}
}

\title{Zadanie 2: Lingwistyczne podsumowania baz danych}

\begin{document}
\maketitle

\section{Cel}
Celem zadania było napisanie aplikacji generującej podsumowania lingwistyczne wybranej bazy danych i przedstawiającej użytkownikowi najlepsze wyniki według zastosowanych miar.

Użytkownik powinien mieć możliwość definiowania kwantyfikatorów, sumaryzatorów i kwalifikatorów oraz możliwość wyboru podsumowywanych atrybutów oraz miar wykorzystywanych do obliczenia jakości podsumowania.

\paragraph{}
Realizacja zadania składała się z 3 etapów:  
\begin{enumerate}
\item Wybór bazy danych lub wygenerowanie z niej widoku posiadającego przynajmniej 10 tysięcy rekordów oraz co najmniej 10 atrybutów dających się podsumować.

\item Implementacja biblioteki obiektowej zawierającej zbiór klas reprezentujących różne rodzaje zbiorów rozmytych typu 1 i 2 i operacji na tych zbiorach. 

\item Generowanie podsumowań dla różnych kwantyfikatorów, kwalifikatorów i sumaryzatorów.
\end{enumerate}


\section{Wprowadzenie}
Poniższy rozdział zawiera wyjaśnienie zagadnień teoretycznych i podstawowych pojęć związanych z wykonywanym zadaniem.

\subsection{Zbiory rozmyte}
Zbiorem rozmytym (ang. \textit{fuzzy set}) $A$ w przestrzeni rozważań $X$ nazywamy zbiór uporządkowanych par postaci:
\begin{equation}
A=\{<x, \mu_A(x)>:x \in X \}
\end{equation}
gdzie $\mu_A:X \rightarrow [0,1] $ jest \textit{funkcją przynależności} , a jej wartość dla elementu $x \in X$ jest \textit{stopniem przynależności} $x$ do zbioru $A$.
\paragraph{}
Uogólnieniem zbiorów rozmytych są zbiory rozmyte typu 2, których podstawą jest rozszerzenie funkcji przynależności o wartościach rzeczywistych do funkcji przynależności typu 2. Przypisuje ona każdemu elementowi z przestrzeni rozważań $X$ zbiór rozmyty typu 1.

Zbiór rozmyty typu 2 jest zdefiniowany następująco:
\begin{equation}
\tilde{A}=\{<x, \mu_{\tilde{A}}(x)>:x \in X \}
\end{equation}
gdzie $\mu_{\tilde{A}}:X \rightarrow F[0,1] $ i $F[0,1]$ jest zbiorem wszystkich zbiorów rozmytych typu 1.

\subsection{Normy trójkątne}
Normy trójkątne są uogólnieniem operacji mnożenia i iloczynu zbiorów rozmytych. Wyróżniamy 2 rodzaje norm trójkątnych:

\begin{enumerate}
\item t-norma - funkcja 2 zmiennych jest t-normą jeżeli jest:
\begin{itemize}
\item przemienna,
\item łączna,
\item monotoniczna,
\item istnieje element neutralny 1;
\end{itemize}
\item s-norma (t-conorma) - funkcja 2 zmiennych jest s-normą jeżeli jest:
\begin{itemize}
\item przemienna,
\item łączna,
\item monotoniczna,
\item istnieje element neutralny 0;
\end{itemize}
\end{enumerate}

\subsection{Podstawowe operacje na zbiorach rozmytych}
\textit{Dopełnieniem} $A^C$ zbioru rozmytego $A$ nazywamy zbiór rozmyty o funkcji przynależności:
\begin{equation}
\mu_{A^C}(x)=1-\mu_A(x)
\end{equation}

\textit{Sumą} zbiorów $A \cup B$ rozmytych nazywamy zbiór rozmyty o funkcji przynależności:
\begin{equation}
\mu_{A \cup B}(x)=\mu_A(x)s\mu_A(x)
\end{equation}
gdzie $s$ jest dowolną s-normą np. operacją maksimum.
\\

\textit{Iloczynem} zbiorów $A \cap B$ rozmytych nazywamy zbiór rozmyty o funkcji przynależności:
\begin{equation}
\mu_{A \cap B}(x)=\mu_A(x)t\mu_A(x)
\end{equation}
gdzie $t$ jest dowolną t-normą np. operacją minimum.
\\

Dopełnieniem $\tilde{A}^C$ zbioru rozmytego typu 2 $\tilde{A}$ nazywamy zbiór rozmyty typu 2 o funkcji przynależności:
\begin{equation}
\mu_{\tilde{A}^C}(x)=\int_{\mu_{\tilde{A}} \in J_x} \mu_x(\mu_{\tilde{A}})/(1-\mu_{\tilde{A}})
 \end{equation}
gdzie $\mu_{\tilde{A}^C}$ jest pierwszorzędnym stopniem przynależności elementu $x$ do zbioru $\tilde{A}$, a $\mu_x(\mu_{\tilde{A}})$ jest drugorzędnym stopniem przynależności.
\\

Suma $\tilde{A} \cup \tilde{B}$ zbiorów rozmytych typu 2 jest zdefiniowana poprzez operację złączenia (ang. \textit{join}) i jest opisana funkcją przynależności:
\begin{equation}
\mu_{\tilde{A} \cup \tilde{B}}(x) = \mu_{\tilde{A}} \sqcup \mu_{\tilde{A}} = \int_{\mu_{\tilde{A}}}\int_{\mu_{\tilde{B}}}(\mu_x(\mu_{\tilde{A}}) t_1 \mu_x(\mu_{\tilde{B}}))/(\mu_{\tilde{A}} s \mu_{\tilde{B}})
\end{equation}
a iloczyn $\tilde{A} \cap \tilde{B}$ jest zdefiniowany poprzez operację \textit{meet} i opisany funkcją przynależności:
\begin{equation}
\mu_{\tilde{A} \cap \tilde{B}}(x) = \mu_{\tilde{A}} \sqcap \mu_{\tilde{A}} = \int_{\mu_{\tilde{A}}}\int_{\mu_{\tilde{B}}}(\mu_x(\mu_{\tilde{A}}) t_1 \mu_x(\mu_{\tilde{B}}))/(\mu_{\tilde{A}} t_2 \mu_{\tilde{B}})
\end{equation}
gdzie $s$ jest dowolną s-normą, a $t_1$ i $t_2$ są dowolnymi t-normami

\subsection{Miary zbiorów rozmytych}
\subsubsection{Liczba kardynalna}
Liczba kardynalna jest podstawową miarą zbiorów rozmytych.

Dla policzalnego zbioru rozmytego A jest to suma stopni przynależności wszystkich elementów z przestrzeni rozważań:
\begin{equation}
|A|=\sum_{x \in X} \mu_A(x)
\end{equation}

Dla zbioru niepoliczalnego jest to całka:
\begin{equation}
|A|=clm(A)=\int_{x \in X} \mu_A(x)\,dx
\end{equation}
\\

W przypadku zbiorów rozmytych typu 2 wzory na kardynalności są następujące:
\\
dla zbioru policzalnego:
\begin{equation}
|\tilde{A}|=\sum_{x \in X} sup\{u \in J_x:\mu_x(u)=1\}
\end{equation}
i niepoliczalnego:
\begin{equation}
|\tilde{A}|=\int_{x \in X} sup\{u \in J_x:\mu_x(u)=1\}\,dx
\end{equation}

\subsubsection{Nośnik}
Nośnikiem $supp(A)$ zbioru rozmytego A jest zbiór klasyczny zawierający wszystkie elementy, które należą do zbioru rozmytego w stopniu większym od zera.  
\\

Dla zbioru rozmytego typu 2 nośnikiem jest zbiór rozmyty typu 1, który zawiera elementy z przestrzeni rozważań $X$ z przypisanymi pierwszorzędnymi funkcjami przynależności, dla których drugorzędne funkcje przynależności są największe.

\subsubsection{Stopień rozmycia}
Stopień rozmycia $in(A)$ zbioru rozmytego A jest określany jako:
\begin{equation}
in(A)=\frac{|supp(A)|}{|X|}
\end{equation}
Im większa jest ta miara, tym zbiór jest bardziej nieprecyzyjny.

\subsection{Lingwistyczne podsumowania baz danych}
\subsubsection{Zmienna lingwistyczna}
Zmienna lingwistyczna $L$ to piątka $<\alpha, H, X, G, K>$, gdzie:
\begin{itemize}
\item $\alpha$ - nazwa zmiennej lingwistycznej
\item $H$ - zbiór wartości lingwistycznych, które należy podać żeby zdefiniować zmienną lingwistyczną - wartości te nazywamy etykietami
\item $X$ - przestrzeń rozważań zbiorów rozmytych, które będą skojarzone z etykietami w zbiorze $H$
\item $G$ - reguła gramatyczna, która pozwala generować etykiety. Najprostszym sposobem jest tutaj wymienienie etykiet
\item $K$ - reguła semantyczna (znaczeniowa) - mówi o tym, że dana etykietę można powiązać z danym zbiorem rozmytym
\end{itemize}

\subsubsection{Podsumowania lingwistyczne według Yagera}
Yager zaproponował podsumowania lingwistyczne następującej postaci:
\begin{itemize}
\item w formie pierwszej: QP jest/ma S [T]
\item w formie drugiej: QP będących/mających W jest/ma S [T]
\end{itemize}
gdzie:
\begin{itemize}
\item P - jest podmiotem podsumowania, czyli obiektem reprezentującym rekord w bazie danych
\item Q - jest to tzw. kwantyfikator i odnosi się do ilości rekordów, których dotyczy podsumowanie. Z kwantyfikatorem jest związana pewna wartość zmiennej lingwistycznej opisana za pomocą zbioru rozmytego. Jeżeli elementy tego zbioru mają wartości z przedziału [0,1] to kwantyfikator nazywamy względnym (relatywnym), a w przeciwnym wypadku bezwzględnym (absolutnym).
\item S i W nazywamy odpowiednio sumaryzatorem i kwalifikatorem. Są one powiązane z wartościami zmiennych lingwistycznych opisujących poszczególne kolumny bazy danych. Kwalifikator i sumaryzator mogą być również złożone z wielu wartości zmiennych lingwistycznych, wtedy są opisane za pomocą iloczynu zbiorów rozmytych związanych z poszczególnymi zmiennymi lingwistycznymi.
\item T - jest miarą prawdziwości podsumowania omówioną w kolejnej sekcji.
\end{itemize}

\paragraph{}
Podsumowania lingwistyczne pozwalają wygenerować za pomocą komputera opis zawartych w bazie danych informacji i przedstawić go w języku naturalnym. Należy jednak najpierw samodzielnie zdefiniować zmienne lingwistyczne i zbiory rozmyte odpowiadające poszczególnym kolumnom w bazie. 

\subsection{Miary jakości podsumowań lingwistycznych}
Miary jakości pozwalają z wszystkich wygenerowanych podsumowań wybrać te, które najlepiej opisują dane zawarte w bazie.

Najpopularniejsze miary zostały opisane poniżej.
\subsubsection{Stopień prawdziwości}
W przypadku kwantyfikatora względnego:
\begin{equation}
T_1 =\mu_Q(\frac{r}{m})
\end{equation}
W przypadku kwantyfikatora absolutnego:
\begin{equation}
T_1 =\mu_Q(r)
\end{equation}

Dla podsumowań w formie pierwszej:
$$r=\sum_{i=1}^m \mu_s(d_i)$$
Dla podsumowań w formie drugiej:
$$r=\frac{\sum_{i=1}^m \mu_s(d_i) \wedge \sum_{i=1}^m \mu_w(d_i)}{\sum_i=1^m \mu_s(d_i)}$$
gdzie:
$\mu_Q, \mu_W, \mu_S$ to funkcje przynależności odpowiednio: kwantyfikatora, kwalifikatora i sumaryzatora, a $d_i$ to i-ty rekord w bazie danych. 

\subsubsection{Stopień nieprecyzyjności}
\begin{equation}
T_2=(\prod_{j=1}^n in(S_j))^{\frac{1}{n}}
\end{equation}
gdzie $S_j$ jest j-tym sumaryzatorem wchodzącym w skład n-elementowego sumaryzatora złożonego.  

\subsubsection{Stopień pokrycia}
\begin{equation}
T_3=\frac{\sum_{i=1}^m t_i}{\sum_{i=1}^m h_i}
\end{equation}
gdzie:\\
\begin{equation}
t_i = \left\{
\begin{tabular}{l l}
1 & \quad \mbox{jeżeli $\mu_s(d_i) > 0$ i $\mu_s(d_i) > 0$}\\
 0 & \quad \mbox{w przeciwnym wypadku}\\ 
\end{tabular} \right .
\end{equation}

\begin{equation}
h_i = \left\{
\begin{tabular}{l l}
1 & \quad \mbox{jeżeli $\mu_s(d_i) > 0$}\\
 0 & \quad \mbox{w przeciwnym wypadku}\\ 
\end{tabular} \right .
\end{equation}

\subsubsection{Miara trafności}


\begin{equation}
T_4=|\prod_{j=1}^n r_j -T_3|
\end{equation}

\begin{equation}
r_j=\frac{\sum_{i=1}^m g_{ij}}{m}
\end{equation}

\begin{equation}
g_{ij} = \left\{
\begin{tabular}{l l}
1 & \quad \mbox{jeżeli $\mu_{s_j}(d_i) > 0$}\\
 0 & \quad \mbox{w przeciwnym wypadku}\\ 
\end{tabular} \right .
\end{equation}
gdzie $i$ jest indeksem rekordu w m-elementowej bazie, a $j$ jest indeksem sumaryzatora wchodzącego w skład n-elementowego sumaryzatora złożonego.

\subsubsection{Długość podsumowania}
\begin{equation}
T_5=2 \cdot (\frac{1}{2})^{|S|}
\end{equation}

gdzie $|S|$ jest liczbą sumaryzatorów wchodzących w skład sumaryzatora złożonego.

\subsubsection{Stopień nieprecyzyjności kwantyfikatora}
\begin{equation}
T_6=1-in(Q)
\end{equation}

\subsubsection{Stopień kardynalności kwantyfikatora}
\begin{equation}
T_7=1-(\frac{|Q|}{|X_Q|})
\end{equation}
gdzie $X_Q$ to przestrzeń rozważań kwantyfikatora $Q$.

\subsubsection{Stopień kardynalności sumaryzatora}
\begin{equation}
T_8=1-|\prod_{j=1}^n\frac{|S_j|}{X_j}|^{\frac{1}{n}}
\end{equation}
gdzie $X_j$ jest przestrzenią rozważań sumaryzatora $S_j$ wchodzącego w skład n-elementowego sumaryzatora złożonego.

\subsubsection{Stopień nieprecyzyjności kwalifikatora}
\begin{equation}
T_9=(\prod_{j=1}^x in(W_j))^{\frac{1}{x}}
\end{equation}
gdzie $W_j$ jest j-tym kwalifikatorem wchodzącym w skład x-elementowego kwalifikatora złożonego.  

\subsubsection{Stopień kardynalności kwalifikatora}
\begin{equation}
T_8=1-|\prod_{j=1}^n\frac{|S_j|}{X_j}|^{\frac{1}{n}}
\end{equation}
gdzie $X_{W_j}$ jest przestrzenią rozważań kwalifikatora $W_j$ wchodzącego w skład x-elementowego kwalifikatora złożonego.

\subsubsection{Długość kwalifikatora}
\begin{equation}
T_11=2 \cdot (\frac{1}{2})^{|W|}
\end{equation}
gdzie $|W|$ jest liczbą kwalifikatorów wchodzących w skład kwalifikatora złożonego.

\subsubsection{Miara optymalna}
Optymalną miarę podsumowania możemy policzyć jako średnią ważoną wyżej wymienionych miar. Najczęściej stosuje się równe wagi dla wszystkich miar, czyli otrzymujemy wtedy średnią arytmetyczną.

\section{Opis implementacji}
Aplikacja została napisana w języku C++ z użyciem frameworka Qt.\\
Składa się z dwóch głównych modułów. Pierwszy z nich stanowi implementację biblioteki klas reprezentujących zbiory rozmyte i operacje na nich oraz lingwistyczne podsumowania baz danych. Drugi moduł jest prostym graficznym interfejsem użytkownika, ale aplikację można również uruchamiać w trybie konsolowym.
\paragraph{}
\subsection{Najważniejsze klasy aplikacji}
Poniżej znajduje się opis najważniejszych klas stworzonych na potrzeby tego zadania:
\begin{enumerate}
\item \textbf{Funkcje przynależności}\\
Klasy te reprezentują funkcje charakterystyczne zbiorów i dziedziczą wspólny interfejs \verb|MembershipFunctionInterface|. Zostały zaimplementowane następujące typy funkcji przynależności:
\begin{enumerate}
\item \verb|TriangleFunction| - funkcja trójkątna,
\item \verb|TrapezoidFunction| - funkcja trapezoidalna, umożliwiające też reprezentowanie też funkcji prostokątnej, a co za tym idzie zbiorów klasycznych (nierozmytych),
\item \verb|DiscreteFunction| - funkcja przynależności przeznaczona do reprezentacji wartości dyskretnych, gdzie dla każdego elementu dziedziny należy jawnie podać jego stopień przynależności.
\end{enumerate}

\item \textbf{Zbiory rozmyte i operacje na zbiorach rozmytych.}\\
Wszystkie poniższe klasy dziedziczą po klasie abstrakcyjnej \verb|FuzzySet| reprezentującej zbiór rozmyty i metody do pobierania stopnia przynależności elementu do zbioru i liczby kardynalnej zbioru:
\begin{enumerate}
\item \verb|BasicFuzzySet| - podstawowa klasa zbioru rozmytego typu 1,
\item \verb|FuzzySetType2| - zbiór rozmyty typu 2,
\item \verb|Union|, \verb|Intersection|, \verb|Complement| - odpowiednio: suma, iloczyn i dopełnienie zbioru rozmytego,
\item \verb|Support|, \verb|AlphaCut| - odpowiednio: nośnik i alfa-przekrój zbioru rozmytego.
\end{enumerate}
\item \textbf{Podsumowania lingwistyczne.}
\begin{enumerate}
\item \verb|LingusisticValue| - wartość zmiennej lingwistycznej - reprezentuje zbiór rozmyty opisujący zmienną lingwistyczną, nazwę zmiennej lingwistycznej i numer kolumny bazy danych, do której się odnosi. Klasa ta reprezentuje w aplikacji kwalifikatory i sumaryzatory.
\item \verb|Quantifier| - kwantyfikator rozszerzający klasę \verb|LingusisticValue|
\item \verb|Summarization| - klasa reprezentująca pojedyncze podsumowania lingwistyczne, zawierająca kwantyfikator, listę sumaryzatorów i listę kwalifikatorów,
\item \verb|SummarizationGenerator| - generator wszystkich możliwych podsumowań (w postaci obiektów \verb|Summarization|) z podanych kwantyfikatorów, klasyfikatorów i sumaryzatorów. Umożliwia zapis podsumowań do pliku.
\item \verb|QualityMeasures| - klasa zawierająca metody służące do liczenia miar jakości podsumowań.
\end{enumerate}

\item \textbf{Klasy narzędziowe.}
\begin{enumerate}
\item \verb|FileParser| - służy do wczytywania następujących danych z plików tekstowych: baza danych, lista kwantyfikatorów oraz lista kwalifikatorów i sumaryzatorów. 
\end{enumerate}

\end{enumerate}

\subsection{Pliki wejściowe i wyjściowe}
Aplikacja korzysta z kilku plików wejściowych:
\begin{enumerate}
\item \verb|database.dat| - plik zawierający bazę danych do podsumowania. Pojedynczy wiersz reprezentuje jeden obiekt, którego atrybuty są oddzielone znakiem tabulacji.
\item \verb|data_description.txt| - opis poszczególnych kolumn bazy danych. Kolejne wiersze pliku odpowiadają kolejnym kolumnom bazy danych i mają następujący format:\\
\verb|typ_danych:nazwa_kolumy|\\
gdzie rozróżniane są 2 typy danych: N - wartość liczbowa, T - wartość tekstowa.
\item \verb|fuzzy_sets_lingvalues.txt| - zawiera opis wartości lingwistycznych, które mogą być kwalifikatorami i sumaryzatorami. Pojedynczy wiersz reprezentują jedną wartość lingwistyczną i ma następujący format:\\
\verb|etykieta:numer_kolumny:typ_funkcji_przynależności:parametry_funkcji|
gdzie mogą występować nastepujące typy funkcji przynależności: \verb|Triangle| - funkcja trójkątna, \verb|TRAPEZOID| - trapezoidalna, \verb|DISCRETE| - dyskretna, 
a parametry funkcji przynależności zależą od jej typu:
\begin{enumerate}
\item dla funkcji trójkątnej: \verb|a:b:c|, gdzie: a - wartość atrybutu dla początku podstawy trójkąta, b - wierzchołek trójkąta, c - koniec podstawy
\item dla funkcji trapezoidalnej: \verb|a:b:c:d|, gdzie a - początek dolnej podstawy, b - początek górnej podstawy, c - koniec górnej podstawy, d - koniec dolnej podstawy. Kiedy a=b i c=d, otrzymujemy funkcję prostokątną.
\item dla funkcji dyskretnej: ciąg par \verb|wartość_atrybutu:stopień_przynależności|
\end{enumerate}

\paragraph{}
Przykłady opisu wartości lingwistycznych:\\
\verb|wysoka temperatura na zewnątrz:4:TRAPEZOID:15:25:40:40|\\
\verb|północny kierunek wiatru:10:DISCRETE:N:1:NNE:0.75:NNW:0.75:NE:0.5:NW:0.5|\\

\item \verb|fuzzy_sets_quantifiers.txt| - zawiera opis kwantyfikatorów używanych do podsumowań. Format pliku jest podobny do opisu wartości lingwistycznych, ale zamiast numeru kolumny, występuje tutaj typ kwantyfikatora:\\
\verb|etykieta:typ_kwantyfikatora:typ_funkcji_przynależności:parametry_funkcji|\\
gdzie typ kwantyfikatora może przyjmować dwie wartości: \verb|RELATIVE| - względny, \verb|ABSOLUTE| - bezwzględny. 
\end{enumerate}

\paragraph{}
Poszczególne linie plików z opisem kwantyfikatorów, kwalifikatorów i sumaryzatorów można tymczasowo ukryć dla aplikacji, za pomocą znaku komentarza \verb|#| na początku linii.

Dodatkowo, do zbiorów rozmytych typu 2 wymagany jest dodatkowy plik z danymi, który określa funkcje przynależności drugiego stopnia. Plik ten, napisany w języku \verb|QtScript|\footnote{Jest to język oparty o \texttt{ECMAScript}, o ten sam standard opiera się język \texttt{JavaScript}.} zawiera tylko jedną funkcję, \verb|fConstructor|, która w wyniku działania zwraca funkcję, która posłuży jako drugorzędna funkcja przynależności.

Funkcja \verb|fConstructor| przyjmuje jako parametry identyfikator funkcji do stworzenia oraz jej wymagane parametry. Otrzymany w wyniku obiekt jest funkcją dwóch argumentów, \textit{x} oraz \textit{y}, pierwszy z nich to wartość pierwszorzędnej funkcji przynależności, natomiast drugi to element dla którego wyznaczane są wartości.

Istnieje przykładowa funkcja trapezoidalna określona za pomocą czterech parametrów: $a$, $b$, $c$ i $d$, opisanych powyżej. Implementacja znajduje się w pliku \verb|func.qs|

\paragraph{}
Wynikiem działania aplikacji jest jeden plik wyjściowy zawierający podsumowania lingwistyczne w postaci tekstowej oraz ich miary jakości. Podsumowania są posortowane malejąco według miar jakości.

\section{Materiały i metody}
Jako baza danych do podsumowania zostały wybrane dane pogodowe ze stacji meteorologicznej znajdującej się w obserwatorium astronomicznym na górze Suchora. Dane te są dostępne w internecie pod adresem:
\url{http://www.as.up.krakow.pl/weather/text/}.

\paragraph{}
Zbiór zawiera obecnie ponad 70 tys. pomiarów, wykonywanych od marca 2008 roku, co pół godziny. Na potrzeby zadania zostało losowo wybrane 10 tys. rekordów.
Każdy pomiar zawiera ponad 20 parametrów, m.in: data, czas, temperatura na zewnątrz, temperatura wewnątrz stacji meteorologicznej, temperatura odczuwalna, prędkość wiatru, kierunek wiatru, wielkość opadów, wilgotność powietrza na zewnątrz, wilgotność powietrza wewnątrz stacji, ciśnienie atmosferyczne. Takie też atrybuty zostały wykorzystane w naszych podsumowaniach z tym, że z daty i czasu został wybrany tylko miesiąc.

\paragraph{}
Dla każdego z podsumowywanych atrybutów został przygotowany został opis przynajmniej 2 wartości lingwistycznych, przedstawiający wartość niską i wysoką. Ponadto został też przygotowany zbiór kwantyfikatorów względnych i bezwzględnych typu: prawie żaden z pomiarów, prawie wszystkie pomiary, około połowa, około 1000.

\paragraph{}
Kolejnym etapem badań było wygenerowane kilka zestawów podsumowań, gdzie na wejściu aplikacji zostały podane zawsze wszystkie kwantyfikatory z wyjątkiem kwantyfikatora "niewiele", ponieważ najwięcej kombinacji atrybutów jest mało prawdopodobnych w rzeczywistości, więc podsumowania z tym kwantyfikatorem zawsze byłyby najwyżej w rankingu.\\
Do podsumowań wykorzystano kilka par atrybutów i dla każdej z nich wykonano 2 pomiary - tylko dla miar podstawowych, czyli T1-T5 oraz dla wszystkich miar, czyli T1-T11. Dla każdego badania przedstawiono 10 najlepszych wyników.

\section{Wyniki}
Poniżej znajdują się otrzymane podczas badań wyniki - w postaci wygenerowanej przez aplikację, bez korekty gramatycznej:

\subsection{Atrybuty: kierunek wiatru, temperatura na zewnątrz - miary T1-T5}

Około 7000 pomiarow ma północny kierunek wiatru. [Quality: 0.5472]\\
Około 1000 pomiarow ma wschodni kierunek wiatru. [Quality: 0.5436]\\
Większość pomiarow ma północny kierunek wiatru. [Quality: 0.524533]\\
Około 3000 pomiarow ma niska temperatura na zewnątrz i północny kierunek wiatru. [Quality: 0.487572]\\
Około 3/4 z pomiarow ma północny kierunek wiatru. [Quality: 0.4736]\\
Około 4000 pomiarow ma niska temperatura na zewnątrz. [Quality: 0.463584]\\
Około 2000 pomiarow ma zachodni kierunek wiatru. [Quality: 0.443]\\
Około 1000 pomiarow ma wysoka temperatura na zewnątrz. [Quality: 0.430672]\\
Około 1/4 z pomiarow ma zachodni kierunek wiatru. [Quality: 0.4215]\\
Około 5000 pomiarow majacych wysoka temperatura na zewnątrz ma tez zachodni kierunek wiatru. [Quality: 0.412171]\\

\subsection{Atrybuty: kierunek wiatru, temperatura na zewnątrz - miary T1-T11}
Około 1000 pomiarow ma wschodni kierunek wiatru. [Quality: 0.689291]\\
Około 1000 pomiarow ma wysoka temperatura na zewnątrz. [Quality: 0.640527]\\
Około 2000 pomiarow ma zachodni kierunek wiatru. [Quality: 0.636759]\\
Około 7000 pomiarow ma północny kierunek wiatru. [Quality: 0.6363]\\
Około 5000 pomiarow ma południowy kierunek wiatru. [Quality: 0.629889]\\
Około 5000 pomiarow ma wysoka temperatura na zewnątrz. [Quality: 0.628853]\\
Około 3000 pomiarow ma południowy kierunek wiatru. [Quality: 0.62762]\\
Około 6000 pomiarow ma południowy kierunek wiatru. [Quality: 0.62762]\\
Około 7000 pomiarow ma południowy kierunek wiatru. [Quality: 0.62762]\\
Około 2000 pomiarow ma południowy kierunek wiatru. [Quality: 0.62762]\\

\subsection{Atrybuty: kierunek wiatru, prędkość wiatru - miary T1-T5}
Około 7000 pomiarow ma północny kierunek wiatru. [Quality: 0.5472]\\
Około 1000 pomiarow ma wschodni kierunek wiatru. [Quality: 0.5436]\\
Większość pomiarow ma północny kierunek wiatru. [Quality: 0.524533]\\
Około 4000 pomiarow ma mała prędkość wiatru i północny kierunek wiatru. [Quality: 0.515102]\\
Około 1000 pomiarow majacych południowy kierunek wiatru ma tez duża prędkość wiatru. [Quality: 0.489416]\\
Prawie wszystkie z pomiarow majacych południowy kierunek wiatru ma tez duża prędkość wiatru. [Quality: 0.489416]\\
Około 1/4 z pomiarow majacych południowy kierunek wiatru ma tez duża prędkość wiatru. [Quality: 0.489416]\\
Około 9000 pomiarow majacych południowy kierunek wiatru ma tez duża prędkość wiatru. [Quality: 0.489416]\\
Około 7000 pomiarow majacych południowy kierunek wiatru ma tez duża prędkość wiatru. [Quality: 0.489416]\\
Około 8000 pomiarow majacych południowy kierunek wiatru ma tez duża prędkość wiatru. [Quality: 0.489416]\\

\subsection{Atrybuty: kierunek wiatru, prędkość wiatru - miary T1-T11}
Około 1000 pomiarow ma wschodni kierunek wiatru. [Quality: 0.689291]\\
Około 5000 pomiarow majacych południowy kierunek wiatru ma tez duża prędkość wiatru. [Quality: 0.646352]\\
Około 8000 pomiarow majacych południowy kierunek wiatru ma tez duża prędkość wiatru. [Quality: 0.644084]\\
Około 2000 pomiarow majacych południowy kierunek wiatru ma tez duża prędkość wiatru. [Quality: 0.644084]\\
Około 7000 pomiarow majacych południowy kierunek wiatru ma tez duża prędkość wiatru. [Quality: 0.644084]\\
Około 6000 pomiarow majacych południowy kierunek wiatru ma tez duża prędkość wiatru. [Quality: 0.644084]\\
Około 3000 pomiarow majacych południowy kierunek wiatru ma tez duża prędkość wiatru. [Quality: 0.644084]\\
Około 4000 pomiarow majacych południowy kierunek wiatru ma tez duża prędkość wiatru. [Quality: 0.644084]\\
Około 9000 pomiarow majacych południowy kierunek wiatru ma tez duża prędkość wiatru. [Quality: 0.644084]\\
Około 1000 pomiarow majacych południowy kierunek wiatru ma tez duża prędkość wiatru. [Quality: 0.644084]\\

\subsection{Atrybuty: wielkość opadów, temperatura na zewnątrz - miary T1-T5}
Większość pomiarow ma niskie opady. [Quality: 0.6]\\
Prawie wszystkie z pomiarow ma niskie opady. [Quality: 0.6]\\
Około 4000 pomiarow ma niska temperatura na zewnątrz. [Quality: 0.463584]\\
Około 1000 pomiarow ma wysoka temperatura na zewnątrz. [Quality: 0.430672]\\
Prawie wszystkie z pomiarow majacych niska temperatura na zewnątrz ma tez niskie opady. [Quality: 0.400298]\\
Około 6000 pomiarow majacych niska temperatura na zewnątrz ma tez niskie opady. [Quality: 0.400298]\\
Około 1000 pomiarow majacych niska temperatura na zewnątrz ma tez niskie opady. [Quality: 0.400298]\\
Około 5000 pomiarow majacych niska temperatura na zewnątrz ma tez niskie opady. [Quality: 0.400298]\\
Około 4000 pomiarow majacych niska temperatura na zewnątrz ma tez niskie opady. [Quality: 0.400298]\\
Około 2000 pomiarow majacych niska temperatura na zewnątrz ma tez niskie opady. [Quality: 0.400298]\\

\subsection{Atrybuty: wielkość opadów, temperatura na zewnątrz - miary T1-T11}
Około 1000 pomiarow ma wysoka temperatura na zewnątrz. [Quality: 0.640527]\\
Około 5000 pomiarow ma wysokie opady. [Quality: 0.634085]\\
Około 7000 pomiarow ma wysokie opady. [Quality: 0.631817]\\
Około 3000 pomiarow ma wysokie opady. [Quality: 0.631817]\\
Około 6000 pomiarow ma wysokie opady. [Quality: 0.631817]\\
Około 4000 pomiarow ma wysokie opady. [Quality: 0.631817]\\
Około 1000 pomiarow ma wysokie opady. [Quality: 0.631817]\\
Około 2000 pomiarow ma wysokie opady. [Quality: 0.631817]\\
Około 9000 pomiarow ma wysokie opady. [Quality: 0.631817]\\
Około 8000 pomiarow ma wysokie opady. [Quality: 0.631817]\\

\subsection{Atrybuty: wielkość opadów, wilgotność powietrza na zewnątrz - miary T1-T5}
Prawie wszystkie z pomiarow ma niskie opady. [Quality: 0.6]\\
Większość pomiarow ma niskie opady. [Quality: 0.6]\\
Większość pomiarow ma wysoka wilgotność powietrza na zewnątrz. [Quality: 0.548875]\\
Około 3/4 z pomiarow ma wysoka wilgotność powietrza na zewnątrz. [Quality: 0.546625]\\
Większość pomiarow ma wysoka wilgotność powietrza na zewnątrz i niskie opady. [Quality: 0.536726]\\
Około 3/4 z pomiarow ma wysoka wilgotność powietrza na zewnątrz i niskie opady. [Quality: 0.532477]\\
Około 7000 pomiarow ma wysoka wilgotność powietrza na zewnątrz. [Quality: 0.50675]\\
Około 7000 pomiarow ma wysoka wilgotność powietrza na zewnątrz i niskie opady. [Quality: 0.501598]\\
Około 4000 pomiarow ma niska temperatura na zewnątrz i wysoka wilgotność powietrza na zewnątrz. [Quality: 0.498673]\\
Około 4000 pomiarow ma niska temperatura na zewnątrz. [Quality: 0.463584]\\

\subsection{Atrybuty: wielkość opadów, wilgotność powietrza na zewnątrz - miary T1-T1}
Około 5000 pomiarow ma wysokie opady. [Quality: 0.634085]\\
Około 5000 pomiarow ma niska wilgotność powietrza na zewnątrz. [Quality: 0.633807]\\
Około 5000 pomiarow majacych wysokie opady ma tez niska wilgotność powietrza na zewnątrz. [Quality: 0.633769]\\
Około 5000 pomiarow majacych niska wilgotność powietrza na zewnątrz ma tez wysokie opady. [Quality: 0.632414]\\
Około 9000 pomiarow ma wysokie opady. [Quality: 0.631817]\\
Około 8000 pomiarow ma wysokie opady. [Quality: 0.631817]\\
Około 4000 pomiarow ma wysokie opady. [Quality: 0.631817]\\
Około 7000 pomiarow ma wysokie opady. [Quality: 0.631817]\\
Około 6000 pomiarow ma wysokie opady. [Quality: 0.631817]\\
Około 3000 pomiarow ma wysokie opady. [Quality: 0.631817]\\

\section{Dyskusja}
Najlepsze wyniki jakości podsumowań dla miar T1-T5 wynoszą około 0,45-0,55, a dla miar T1-T11 są wyraźnie wyższe i wynoszą około 0,6-0,65. Pomimo to, przyglądając się wygenerowanym podsumowaniom, to te pierwsze wydają się być bliższe prawdy.

\paragraph{}
Dla miar T1-T11 często wartości jakości podsumowania są takie same nawet dla sprzecznych podsumowań. Widać tutaj wyraźnie wpływ miar T6 i T7, które dają takie same wyniki dla większości zastosowanych przez nas kwantyfikatorów. Ponadto część miar nie wykorzystuje wartości danych zawartych w podsumowywanej bazie, a jedynie własności zbiorów rozmytych opisujących zmienne lingwistyczne.

\paragraph{}
Podczas testów zauważyliśmy też, że nie ma sensu generowanie podsumowań dla wielu kolumn jednocześnie. Im więcej atrybutów w podsumowaniu, tym mniejsze prawdopodobieństwo jego prawdziwości - poza kwantyfikatorami typu "niewiele", "prawie żaden", które w takiej sytuacji zdecydowanie dominują w zbiorze najlepszych podsumowań. Ponadto czas obliczeń rośnie tutaj wykładniczo wraz ze wzrostem liczby podsumowywanych atrybutów. Dla 2 atrybutów, zawierających 2-3 wartości zmiennych lingwistycznych i 13 kwantyfikatorów jest do policzenia średnio 300 podsumowań, co zajmuje kilka sekund, dla 5 kolumn jest to już 30-40 tys. podsumowań i czas do 10 minut. Dla 10 atrybutów liczba podsumowań rośnie do kilkudziesięciu milionów, a czas obliczeń wyniósłby kilkaset godzin.

\paragraph{}
Podsumowania lingwistyczne są narzędziem ułatwiającym analizę wielkich zbiorów danych, których przejrzenie i podsumowanie przez człowieka byłoby niemożliwe. Nie wystarczy tutaj jednak posiadanie samej biblioeki/aplikacji do podsumowań lingwistycznych. Należy również posiadać pewną wiedzę na temat zbiorów rozmytych oraz miar jakości podsumowań. Odpowiedni dobór zmiennych lingwistycznych i ich opis za pomocą zbiorów rozmytych oraz wybór miar jakości mają tutaj zasadnicze znaczenie.


\section{Wnioski}
\begin{itemize}
 \item Podsumowania lingwistyczne znacznie ułatwiają opis dużych zbiorów danych, które byłyby niemożliwe do przejrzenia i podsumowania przez człowieka.
 \item Duży wpływ na otrzymane wyniki ma dobór odpowiednich miar jakości. Zastosowanie wszystkich miar z takimi samymi wagami powoduje uzyskanie zbliżonych wyników dla zupełnie różnych podsumowań.
 \item W przypadku równych wag dla miar jakości lepiej sprawdzają się miary podstawowe - T1-T5 niż zestaw wszystkich miar T1-T11.
 \item Miary jakości nie uwzględniające wartości z bazy danych, a tylko własności opisujących je zbiorów rozmytych, mogą znacznie zaburzyć otrzymane wyniki - np. miary T6 i T7, które dają takie same wyniki dla 9 z 14 naszych kwantyfikatorów.
 \item Generowanie podsumowań dla wielu atrybutów jednocześnie jest bardzo wymagające obliczeniowo, a wnosi mało informacji - prawie wszystkie wygenerowane kombinacje sumaryzatorów i kwalifikatorów będą spełnione tylko dla niewielkiej liczby lub żadnego z rekordów w bazie danych.
\end{itemize}

\begin{thebibliography}{0}
\bibitem .A. Niewiadomski, \textsl{Methods for the Linguistic Summarization of Data: Applications of Fuzzy Stes and Their Extensions}, Akademicka Oficyna Wydawnicza EXIT , Warszawa 2008
	
\end{thebibliography}
\end{document}
